\documentclass[conference]{IEEEtran}
\IEEEoverridecommandlockouts

\usepackage{cite}
\usepackage{amsmath,amssymb,amsfonts}
\usepackage{algorithmic}
\usepackage{graphicx}
\usepackage{textcomp}
\usepackage{xcolor}

\def\BibTeX{{\rm B\kern-.05em{\sc i\kern-.025em b}\kern-.08em
T\kern-.1667em\lower.7ex\hbox{E}\kern-.125emX}}

\begin{document}

\title{Measuring Properties of a CCD Spectrograph}

\author{\IEEEauthorblockN{1\textsuperscript{st} Kryštof Havránek}
\IEEEauthorblockA{\textit{Course: Space Engineering \the\year} \\
\textit{Czech Technical University in Prague}\\
Technicka 2, Prague, Czech Republic \\
havrakry@fel.cvut.cz}
\and
\IEEEauthorblockN{2\textsuperscript{nd} Vega Grau Esteve}
\IEEEauthorblockA{\textit{Course: Space Engineering \the\year} \\
\textit{Politechnical University of Valencia}\\
Camino de Vera 46022, Valencia, Spain \\
vega03upv@gmail.com}
\and
\IEEEauthorblockN{Mentor}
\IEEEauthorblockA{\textit{Course: Space Engineering \the\year} \\
\textit{Astronomical Institute of the CAS}\\
Fričova 298, Ondřejov, Czech Republic \\
martin.jelinek@asu.cas.cz}
}

\maketitle

\begin{abstract}
Following paper concerns itself with a process of measuring and calculating parameters of a Cloude CCD spectrograph connected to a Perek 2 meter optical telescope. Measured properties were readout noise, dark current behavior and lastly gain.
\end{abstract}

%------------------------------------------------------------

\section{INTRODUCTION}

One of the primary tools used to investigate the physical properties of celestial objects is astronomical spectroscopy. This carries out a process that enables the determination of stellar temperatures, chemical composition, radial velocities and other fundamental parameters by dispersing incoming light into its constituent wavelengths. Nowadays, modern spectroscopic observations rely on large-aperture telescopes which are combined with sensitive CCD detectors and stable spectrographs.

As emphasized by Martinez et al. in A Practical Guide to CCD Astronomy, the scientific quality of CCD-based observations depends not only on the optical system but also on a precise understanding of detector behavior and noise sources, as well as on proper calibration procedures \cite{Martinez2001}. Similarly, Lessons from the Masters highlights that accurate spectroscopic measurements require careful instrument characterization and calibration before any scientific interpretation can be performed \cite{Moore2014}.

This protocol describes the instrumental setup and theoretical background relevant to spectroscopic observations carried out with the Ondřejov 2-m telescope. Particular emphasis is placed on the principles of spectrograph operation, CCD detector characteristics, and the main noise sources affecting the acquired data. This theoretical foundation provides the basis for the data reduction and analysis steps described in later sections.


%------------------------------------------------------------

\section{ONDŘEJOV 2-M TELESCOPE}

The Ondřejov 2-m telescope, also known as the Perek telescope, is built as a classical Cassegrain reflector with a primary mirror which has a diameter of two meters and is operated by the Astronomical Institute of the Czech Academy of Sciences. The telescope provides a combination of high light-gathering power and mechanical stability, making it well suited for high-resolution spectroscopic observations of relatively faint astronomical targets, as said in \cite{Ondrejov2m}.

In a Cassegrain optical configuration, the light that is collected by the primary mirror is reflected towards a secondary mirror and then redirected to the focal plane through a central aperture. In addition to the Cassegrain focus, the Ondřejov 2-m telescope is equipped with a coudé focus, which redirects the light path through a series of mirrors to a fixed laboratory location. This configuration allows spectrographs and auxiliary instrumentation to be mounted in a stationary and mechanically stable environment, significantly reducing the effects of flexure and improving long-term stability.

For this project, the spectroscopic observations were carried out using the coudé spectrograph with a focal length of 700\,mm. The coudé spectrograph at the Ondřejov Observatory does not employ a cross-dispersing element to spatially separate diffraction orders on the detector, contrary to what happens in cross-dispersed echelle spectrographs. Instead, what it does is select different spectral regions and different diffraction orders using optical filters placed in the optical path. As stated in \cite{Ondrejov2m}, this approach simplifies the optical layout while allowing targeted observations in specific wavelength ranges.

This telescope is equipped with accurate guiding and tracking systems, which are essential for spectroscopic observations performed with narrow entrance slits or optical fibers. The systems implemented in the Ondřejov 2-m telescope ensure stable target positioning during long exposures. Guiding errors can directly lead to reduced signal-to-noise ratio or artificial broadening of spectral features, directly  affecting the quality of the recorded spectra, as it is said in \cite{Moore2014}.

\begin{figure}[htbp]
\centering
\includegraphics[width=0.95\columnwidth]{../im/im1.jpg}
\caption{Ondřejov 2-m Telescope}
\label{fig:im1}
\end{figure}

\begin{figure}[htbp]
\centering
\includegraphics[width=0.95\columnwidth]{../im/im2.jpg}
\caption{Dome of the telescope}
\label{fig:im2}
\end{figure}

%------------------------------------------------------------

\section{SPECTROGRAPH OVERVIEW AND OPTICAL LAYOUT}

To start with, a spectrograph is an optical instrument designed in a way that incoming light is dispersed into its different wavelength components. In the field of astronomical spectrographs, this kind of dispersion is achieved by using a diffraction grating. According to \cite{Moore2014}, basic components of the spectrograph that will give the expected result are an entrance slit or fiber, a collimator, a dispersive element, a camera system and a detector.

To continue with, the working mechanism of the spectrograph begins with the light that enters the spectrograph through the slit or optical fiber, collimating into a parallel beam. In this beam, the diffraction grating is encountered, which separates the light according to wavelength. Next, the dispersed light is subsequently focused onto the CCD detector by camera optics, where the spectrum will be recorded. As mentioned in \cite{Martinez2001}, fiber-fed spectrographs are often preferred because they decouple the spectrograph mechanically from the telescope, reducing flexure and improving long-term stability.

However, fiber-fed systems introduce additional considerations. These considerations include variations in fiber positioning or illumination, which can lead to changes in intensity across the detector. The changes in intensity will have to be corrected during flat-field calibration. In \cite{Martinez2001}, the importance of careful flat-fielding and illumination control is mentioned as an imperfect illumination of the spectrograph entrance can imprint systematic patterns onto the recorded spectra.

\begin{figure}[htbp]
\centering
\includegraphics[width=0.95\columnwidth]{../im/im3.png}
\caption{Optical layout of a coudé spectrograph}
\label{fig:im3}
\end{figure}

\begin{figure}[htbp]
\centering
\includegraphics[width=0.95\columnwidth]{../im/im4.png}
\caption{Schematic illustration of a telescope optical layout}
\label{fig:im4}
\end{figure}

%------------------------------------------------------------

\section{SPECTRAL ORDERS AND WAVELENGTH CALIBRATION}

In order to do a good wavelength calibration and choose the correct spectral orders, it is to be understood that high-resolution spectrographs often operate in multiple diffraction orders, more precisely, when echelle gratings are used. In this system, each spectral order corresponds to a different diffraction condition of the grating. This allows a wider wavelength range to be recorded simultaneously on the CCD detector. In \cite{Moore2014} it is described that in order to separate overlapping orders, a cross-dispersing element is employed, producing a two-dimensional spectral format.

To get a good spectroscopic analysis, it is essential to have an accurate wavelength calibration. The level of accuracy needed in the calibration is usually achieved by using emission-line lamps, such as Thorium–Argon (ThAr) lamps, which provide a dense set of narrow spectral lines with well-known wavelengths. As described in \cite{Martinez2001}, ThAr lamps are very suitable for high-precision wavelength calibration due to their stability and line density.

Even so, while the detailed matching of observed ThAr spectra to reference line lists is part of the data reduction process, an understanding of the underlying calibration principle is necessary when designing the observational protocol and interpreting the results.

\begin{figure}[htbp]
\centering
\includegraphics[width=0.95\columnwidth]{../im/im5.jpeg}
\caption{Diffraction orders produced by a grating for multiple wavelengths.}
\label{fig:im5}
\end{figure}

\begin{figure}[htbp]
\centering
\includegraphics[width=0.95\columnwidth]{../im/im6.jpg}
\caption{Diffraction pattern showing the central maximum and higher-order diffraction fringes}
\label{fig:im6}
\end{figure}

%------------------------------------------------------------

\section{CCD DETECTOR FUNDAMENTALS}

Charge-Couple Devices are widely used in astronomical spectroscopy since they are well suited for the detection of faint astronomical signals and for applications requiring precise spectroscopic and photometric measurements. This is because of their properties, which include high quantum efficiency, linear response over a wide dynamic range and relatively low noise characteristics.

The operating mode of these devices is that incident photons interact with the silicon substrate, generating photoelectrons through the photoelectric effect. This way, the number of generated electrons is proportional to the incident photon flux and the detector quantum efficiency. The generated photoelectrons are accumulated in potential wells associated with individual pixels during the exposure time. Then, using a sequence of clocked voltage shifts, the stored charge is transferred across the detector and eventually read out at an amplifier. In this readout process, the collected charge is converted to a voltage signal in order to be digitized by analog-to-digital converter (ADC) and recorded in units of Analog-to-Digital Units (ADU). During this process, several instrumental effects that have to be characterized and corrected during data reduction are introduced \cite{Moore2014}. This is expressed as:

\begin{equation}
N_{e^-} = G \cdot S_{\text{ADU}},
\label{eq:gain_def}
\end{equation}

Then, the CCD gain determines the conversion between the number of electrons and the recorded ADU value, in units of electrons per ADU. Knowledge of this gain is essential for converting measured signals into physically meaningful quantities and for correctly estimating the contributions of different noise sources. In \cite{Martinez2001}, it is emphasized that an incorrect gain value leads to systematic errors in noise estimations and subsequently in signal-to-noise ratio calculations, affecting the reliability of scientific results.

Another important characteristic of CCD detectors is their linear response, which assures that the recorded signal is proportional to the number of incident photons over a wide range of illumination levels. The assumption of linearity is crucial for many calibration procedures, such as flat-field correction and gain determination. Deviations from linearity may occur at very low signal levels due to electronic effects or near full-well capacity due to charge saturation. Consequently, linearity tests are commonly performed to verify the operational range of the detector \cite{Martinez2001}.

In addition to gain and linearity, CCD performance is influenced by pixel-to-pixel variations in sensitivity, commonly referred to as pixel response non-uniformity. These variations arise from manufacturing imperfections and are corrected using flat-field calibration frames. Understanding the physical operation of CCD detectors and their fundamental characteristics is therefore essential for the correct interpretation of spectroscopic data and forms the basis for the calibration and noise analysis procedures described in subsequent sections.

%------------------------------------------------------------

\section{NOISE SOURCES IN CCD DETECTORS}

Several independent noise sources affect measurements performed with CCD detectors. The main contributions are read noise, dark current noise, and photon (shot) noise. A clear understanding of the physical origin and statistical properties of these noise sources is essential for interpreting CCD data and for optimizing observational strategies.

To start understanding, there is read noise, which is introduced during the electronic readout process of the CCD and is independent of the exposure time. It comes from multiple electronic components, such as the output amplifier, the charge-to-voltage conversion stage or the analog-to-digital converter. Fluctuations in the process of the readout lead to uncertainty in the measured signal, even in the absence of incident light.

Read noise is typically characterized using bias (zero-exposure) frames, which record the electronic offset and readout fluctuations of the detector. By analyzing the statistical dispersion of pixel values across multiple bias frames, the read noise can be estimated on a pixel-by-pixel basis or as a global detector parameter. As it can be seen in \cite{Moore2014}, read noise limits low signal levels, where it can dominate over other noise sources and significantly reduce the signal-to-noise ratio. This happens because read noise does not depend on exposure time, as the collected signal increases, its relative contribution decreases. To achieve reliable measurements, read noise has to be minimized for short exposures or observations of faint sources. Read noise is well described by a Gaussian (normal) distribution, as it originates from the sum of many independent electronic processes in the readout chain \cite{Moore2014}.

\begin{equation}
\sigma_{\text{read}}(x,y)
=
\sqrt{
\frac{1}{N-1}
\sum_{i=1}^{N}
\left[
B_i(x,y) - \langle B(x,y) \rangle
\right]^2
},
\label{eq:read_noise_pixel}
\end{equation}

There is also dark current noise, which arises from thermally generated electrons within the silicon lattice of the CCD, even when no light is incident on the detector. These electrons generate because of thermal excitation and are indistinguishable from the photoelectrons that come from incident photons. As a result, dark current introduces an additional signal that accumulates linearly with exposure time. Since the generation of thermal electrons is a random process, the associated dark noise follows Poisson statistics. The standard deviation of the dark noise is therefore:

\begin{equation}
\sigma_{\text{dark}} = \sqrt{N_{\text{dark}}}
= \sqrt{I_{\text{dark}}\,t},
\label{eq:dark_noise}
\end{equation}

As stated in \cite{Martinez2001}, dark current strongly depends on detector temperature and can increase rapidly with the slightest temperature changes. To minimize this, CCD detectors are commonly cooled, avoiding the temperature changes and reducing the dark current and associated noise. In detectors that have a good cooling system, the contribution of dark current noise may become negligible compared to other noise sources. Dark current noise follows Poisson statistics, since it is generated by random thermal electron production events within the CCD lattice \cite{Martinez2001}.

Lastly, the other noise source that can be found is the photon noise, also known as shot noise, which originates from the statistical nature of photon arrival at the detector. The number of detected photons fluctuates from one exposure to another even for a constant light source, since they arrive at the CCD detector following Poisson statistics. The uncertainty these fluctuations contribute to the detected signal cannot be eliminated by calibration. If photons are detected during the exposure, the corresponding shot noise is given by:

\begin{equation}
\sigma_{\text{shot}} = \sqrt{N_{\text{ph}}}.
\label{eq:shot_noise}
\end{equation}


Photon noise becomes the dominant noise source at high fluxes since it increases with signal level. Compared to read noise and dark current noise, photon noise represents a fundamental physical limit and cannot be reduced by just making instrumental improvements. The way to reduce its relative impact would be by increasing the signal level, as stated in \cite{Martinez2001}, where it is stated that it can be reduced through longer exposure times or larger telescope apertures. Photon noise also follows a Poisson distribution, reflecting the stochastic nature of photon arrival at the detector \cite{Martinez2001}.

Finally, these different noise sources must be considered together. Assuming the noise sources are statistically independent, the total noise in a CCD pixel is given by the quadratic sum of all contributions:

\begin{equation}
\sigma_{\text{total}}^{2}
=
\sigma_{\text{read}}^{2}
+
\sigma_{\text{dark}}^{2}
+
\sigma_{\text{shot}}^{2}.
\label{eq:total_noise}
\end{equation}


This expression provides a practical framework for evaluating detector performance and for optimizing exposure times in spectroscopic observations. As it can be seen in both \cite{Martinez2001, Moore2014}, understanding which noise source dominates under given observational conditions is essential for designing efficient observing strategies and for interpreting the measured data.

\begin{figure}[htbp]
\centering
\includegraphics[width=0.95\columnwidth]{../im/im7.png}
\caption{Overview of signal formation and noise sources in a CCD system.}
\label{fig:im7}
\end{figure}

\begin{figure}[htbp]
\centering
\includegraphics[width=0.95\columnwidth]{../im/im8.gif}
\caption{Dark current dependence on CCD temperature.}
\label{fig:im8}
\end{figure}

\begin{figure}[htbp]
\centering
\includegraphics[width=0.95\columnwidth]{../im/im9.png}
\caption{Read noise as a function of CCD readout frequency.}
\label{fig:im9}
\end{figure}

%------------------------------------------------------------

\section{CALIBRATION FRAMES}

Calibration frames are required in any configuration of the CCD detectors since there are instrumental effects present in CCD data. These different types of calibration frames are essential to separate the true astronomical signal from the detector and instrumental related contributions. Different contributions are needed to address different aspects of the detector response. The most used ones are bias frames, dark frames and flat-field frames.

Going into more depth of these types of calibration frames, bias or zero-exposure frames are used to measure the electronic offset introduced during the readout process and to characterize the read noise of the detector. Since bias frames are acquired with no light incident on the CCD, they contain only the electronic bias level and readout fluctuations. As stated in \cite{Moore2014}, by combining multiple bias frames, a master bias frame can be constructed and subsequently subtracted from all other images to remove this electronic offset. This is represented as:

\begin{equation}
I_{\text{bias-corrected}} = I_{\text{raw}} - B_{\text{master}}.
\label{eq:bias_subtraction}
\end{equation}

Next, dark frames are acquired with the same exposure time and detector temperature as the science observations but without any incident light. They are used to characterize the dark current and its associated noise. The accumulated dark signal increases linearly with exposure time and depends strongly on detector temperature. From \cite{Martinez2001}, dark frames can also capture fixed-pattern components of the dark signal, such as hot pixels. These can be removed by subtracting a master dark frame from the data.

Finally, flat-field frames are used to correct for pixel-to-pixel sensitivity variations and large-scale illumination gradients across the detector. In perfect conditions, flat-field frames would require a uniform illumination of the CCD, but in practice, this perfect uniformity is next to impossible to achieve. In fiber-fed spectrographs, variations in fiber positioning or illumination can introduce additional gradients in the flat-field images. As stated in \cite{Martinez2001}, these effects must be carefully accounted for during calibration to avoid introducing systematic errors into the corrected spectra. They are corrected following the next equation:

\begin{equation}
I_{\text{flat-corrected}} =
\frac{I_{\text{raw}} - B_{\text{master}} - D_{\text{master}}}{F_{\text{norm}}},
\quad
F_{\text{norm}} = \frac{F_{\text{master}}}{\left\langle F_{\text{master}} \right\rangle}.
\label{eq:flat_correction}
\end{equation}

\begin{figure}[htbp]
\centering
\includegraphics[width=0.95\columnwidth]{../im/im10.jpg}
\caption{Mathematical calibration flow.}
\label{fig:im10}
\end{figure}

\begin{figure}[htbp]
\centering
\includegraphics[width=0.95\columnwidth]{../im/im11.jpg}
\caption{Practical calibration workflow.}
\label{fig:im11}
\end{figure}

%------------------------------------------------------------

\section*{Acknowledgment}

The authors would like to thank the Astronomical Institute of the Czech Academy of Sciences for providing access to the Ondřejov Observatory facilities and for their support during the measurements.

\bibliographystyle{unsrt}
\bibliography{bibliography}

\end{document}
