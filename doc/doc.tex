\documentclass[conference]{IEEEtran}
\IEEEoverridecommandlockouts
% The preceding line is only needed to identify funding in the first footnote. If that is unneeded, please comment it out.
\usepackage{cite}
\usepackage{amsmath,amssymb,amsfonts}
\usepackage{algorithmic}
\usepackage{graphicx}
\usepackage{textcomp}
\usepackage{xcolor}
\def\BibTeX{{\rm B\kern-.05em{\sc i\kern-.025em b}\kern-.08em
    T\kern-.1667em\lower.7ex\hbox{E}\kern-.125emX}}
\begin{document}

\title{Measuring Properties of a CCD spectrograph}

\author{\IEEEauthorblockN{1\textsuperscript{st} Kryštof Havránek}
\IEEEauthorblockA{\textit{Course: Space Engineering \the\year} \\
\textit{Czech Technical University in Prague}\\
Technicka 2, Prague, Czech Republic \\
havrakry@fel.cvut.cz}
\and
\IEEEauthorblockN{2\textsuperscript{nd} Esteve Vega Grau}
\IEEEauthorblockA{\textit{Course: Space Engineering \the\year} \\
\textit{Technical University of Valencia}\\
Camino de Vera 46022, Prague, Spain \\
email address}
\and
\IEEEauthorblockN{Mentor}
\IEEEauthorblockA{\textit{Course: Space Engineering \the\year} \\
\textit{Astronomical Institute of the CAS}\\
Fričova 298, Ondřejov, Czech Republic \\
martin.jelinek@asu.cas.cz}
}

\maketitle

\begin{abstract}
Following paper concerns itself with a process of measuring and calculating parameters of a Cloude CCD spectrograph connected to a Perek 2 meter optical telescope. Measured properties were readout noise, dark current behavior and lastly gain.
\end{abstract}


\section{Introduction}


\section*{Acknowledgment}

\section*{References}


\bibliographystyle{unsrt}
\bibliography{bibliography}

\end{document}
