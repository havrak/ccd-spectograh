\documentclass[conference]{IEEEtran}
\IEEEoverridecommandlockouts

\usepackage{cite}
\usepackage{amsmath,amssymb,amsfonts}
\usepackage{algorithmic}
\usepackage{graphicx}
\usepackage{textcomp}
\usepackage{xcolor}

\def\BibTeX{{\rm B\kern-.05em{\sc i\kern-.025em b}\kern-.08em
T\kern-.1667em\lower.7ex\hbox{E}\kern-.125emX}}

\begin{document}

\title{Measuring Properties of a CCD Spectrograph}

\author{\IEEEauthorblockN{1\textsuperscript{st} Kryštof Havránek}
\IEEEauthorblockA{\textit{Course: Space Engineering \the\year} \\
\textit{Czech Technical University in Prague}\\
Technicka 2, Prague, Czech Republic \\
havrakry@fel.cvut.cz}
\and
\IEEEauthorblockN{2\textsuperscript{nd} Vega Grau Esteve}
\IEEEauthorblockA{\textit{Course: Space Engineering \the\year} \\
\textit{Politechnical University of Valencia}\\
Camino de Vera 46022, Valencia, Spain \\
vega03upv@gmail.com}
\and
\IEEEauthorblockN{Mentor}
\IEEEauthorblockA{\textit{Course: Space Engineering \the\year} \\
\textit{Astronomical Institute of the CAS}\\
Fričova 298, Ondřejov, Czech Republic \\
martin.jelinek@asu.cas.cz}
}

\maketitle

\begin{abstract}
Following paper concerns itself with a process of measuring and calculating parameters of a Cloude CCD spectrograph connected to a Perek 2 meter optical telescope. Measured properties were readout noise, dark current behavior and lastly gain.
\end{abstract}

%------------------------------------------------------------

\section{Introduction}

One of the primary tools used to investigate the physical properties of celestial objects is astronomical spectroscopy. This carries out a process that enables the determination of stellar temperatures, chemical composition, radial velocities and other fundamental parameters by dispersing incoming light into its constituent wavelengths. Nowadays, modern spectroscopic observations rely on large-aperture telescopes which are combined with sensitive CCD detectors and stable spectrographs.

As emphasized by Martinez et al.\ in \textit{A Practical Guide to CCD Astronomy}, the scientific quality of CCD-based observations depends not only on the optical system but also on a precise understanding of detector behavior and noise sources, as well as on proper calibration procedures \cite{Martinez2001}. Similarly, \textit{Lessons from the Masters} highlights that accurate spectroscopic measurements require careful instrument characterization and calibration before any scientific interpretation can be performed \cite{Moore2014}.

This protocol describes the instrumental setup and theoretical background relevant to spectroscopic observations carried out with the Ondřejov 2-m telescope. Particular emphasis is placed on the principles of spectrograph operation, CCD detector characteristics, and the main noise sources affecting the acquired data. This theoretical foundation provides the basis for the data reduction and analysis steps described in later sections.

%------------------------------------------------------------

\section{Ondřejov 2-m Telescope}

The Ondřejov 2-m telescope, also known as the Perek telescope, is built as a classical Cassegrain reflector with a primary mirror which has a diameter of two meters and is operated by the Astronomical Institute of the Czech Academy of Sciences. The telescope provides a combination of high light-gathering power and mechanical stability, making it well suited for high-resolution spectroscopic observations of relatively faint astronomical targets, as said in \cite{Ondrejov2m}.

In a Cassegrain optical configuration, the light that is collected by the primary mirror is reflected towards a secondary mirror and then redirected to the focal plane through a central aperture. In addition to the Cassegrain focus, the Ondřejov 2-m telescope is equipped with a coudé focus, which redirects the light path through a series of mirrors to a fixed laboratory location. This configuration allows spectrographs and auxiliary instrumentation to be mounted in a stationary and mechanically stable environment, significantly reducing the effects of flexure and improving long-term stability.

For this project, the spectroscopic observations were carried out using the coudé spectrograph with a focal length of 700\,mm. The coudé spectrograph at the Ondřejov Observatory does not employ a cross-dispersing element to spatially separate diffraction orders on the detector, contrary to what happens in cross-dispersed echelle spectrographs. Instead, what it does is select different spectral regions and different diffraction orders using optical filters placed in the optical path. As stated in \cite{Ondrejov2m}, this approach simplifies the optical layout while allowing targeted observations in specific wavelength ranges.

This telescope is equipped with accurate guiding and tracking systems, which are essential for spectroscopic observations performed with narrow entrance slits or optical fibers. The systems implemented in the Ondřejov 2-m telescope ensure stable target positioning during long exposures. Guiding errors can directly lead to reduced signal-to-noise ratio or artificial broadening of spectral features, directly  affecting the quality of the recorded spectra, as it is said in \cite{Moore2014}.

%------------------------------------------------------------

\section{Spectrograph Overview and Optical Layout}

A spectrograph is an optical instrument designed so that incoming light is dispersed into its different wavelength components. In astronomical spectrographs, this dispersion is achieved by using a diffraction grating. According to \cite{Moore2014}, the basic components required for a spectrograph to function correctly are an entrance slit or optical fiber, a collimator, a dispersive element, a camera system, and a detector.

The working mechanism of the spectrograph begins with the light entering through the slit or optical fiber, where it is collimated into a parallel beam. This beam encounters the diffraction grating, which separates the light according to wavelength. The dispersed light is then focused onto the CCD detector by camera optics, where the spectrum is recorded. As mentioned in \cite{Martinez2001}, fiber-fed spectrographs are often preferred because they decouple the spectrograph mechanically from the telescope, reducing flexure and improving long-term stability.

However, fiber-fed systems introduce additional considerations. Variations in fiber positioning or illumination can lead to changes in intensity across the detector, which must be corrected during flat-field calibration. As stated in \cite{Martinez2001}, imperfect illumination of the spectrograph entrance can imprint systematic patterns onto the recorded spectra, highlighting the importance of careful flat-fielding and illumination control.

%------------------------------------------------------------

\section{Spectral Orders and Wavelength Calibration}

High-resolution spectrographs often operate in multiple diffraction orders, particularly when echelle gratings are used. In this system, each spectral order corresponds to a different diffraction condition of the grating, allowing a wide wavelength range to be recorded simultaneously on the CCD detector. As described in \cite{Moore2014}, overlapping orders are separated using a cross-dispersing element, producing a two-dimensional spectral format.

Accurate wavelength calibration is essential for spectroscopic analysis. This is typically achieved using emission-line lamps, such as Thorium--Argon (ThAr) lamps, which provide a dense set of narrow spectral lines with well-known wavelengths. As described in \cite{Martinez2001}, ThAr lamps are particularly suitable for high-precision wavelength calibration due to their stability and line density.

While the detailed matching of observed ThAr spectra to reference line lists is part of the data reduction process, understanding the underlying calibration principle is essential when designing the observational protocol and interpreting the results.

%------------------------------------------------------------

\section{CCD Detector Fundamentals}

Charge-coupled devices are widely used in astronomical spectroscopy due to their high quantum efficiency, linear response over a wide dynamic range, and relatively low noise characteristics. These properties make CCDs suitable for the detection of faint astronomical signals and for applications requiring precise spectroscopic and photometric measurements.

Incident photons interact with the silicon substrate of the CCD, generating photoelectrons through the photoelectric effect. The number of generated electrons is proportional to the incident photon flux and the detector quantum efficiency. The photoelectrons are accumulated in potential wells associated with individual pixels during the exposure time. Through a sequence of clocked voltage shifts, the stored charge is transferred across the detector and eventually read out at an amplifier.

During readout, the collected charge is converted into a voltage signal and digitized by an analog-to-digital converter, producing values recorded in Analog-to-Digital Units (ADU). This process introduces several instrumental effects that must be characterized and corrected during data reduction \cite{Moore2014}.

The CCD gain determines the conversion between the number of electrons and the recorded ADU value. Accurate knowledge of the gain is essential for converting measured signals into physically meaningful quantities and for estimating noise contributions. As emphasized in \cite{Martinez2001}, an incorrect gain value leads to systematic errors in noise estimation and signal-to-noise ratio calculations.

Another important characteristic of CCD detectors is their linear response, which ensures that the recorded signal is proportional to the number of incident photons over a wide range of illumination levels. Deviations from linearity may occur at very low signal levels or near full-well capacity due to charge saturation. Linearity tests are therefore commonly performed to determine the valid operational range of the detector \cite{Martinez2001}.

%------------------------------------------------------------

\section{Noise Sources in CCD Detectors}

Several independent noise sources affect measurements performed with CCD detectors. The main contributions are read noise, dark current noise, and photon (shot) noise.

Read noise is introduced during the electronic readout process and is independent of exposure time. It originates from multiple electronic components such as the output amplifier, charge-to-voltage conversion stage, and analog-to-digital converter. Fluctuations during readout lead to uncertainty in the measured signal, even in the absence of incident light. Read noise is typically characterized using bias frames and dominates at low signal levels \cite{Moore2014}.

Dark current noise arises from thermally generated electrons within the silicon lattice of the CCD. These electrons are indistinguishable from photoelectrons and accumulate linearly with exposure time. Since the generation of thermal electrons is a random process, the associated dark noise follows Poisson statistics.
% TODO: Insert dark noise equation here

As stated in \cite{Martinez2001}, dark current strongly depends on detector temperature and can increase rapidly with small temperature variations. CCD detectors are therefore commonly cooled to reduce dark current and its associated noise.

Photon noise, also known as shot noise, originates from the statistical nature of photon arrival at the detector. Even for a constant light source, the number of detected photons fluctuates according to Poisson statistics, introducing an uncertainty that cannot be removed by calibration.
% TODO: Insert photon noise equation here

Assuming that the noise sources are statistically independent, the total noise in a CCD pixel is given by the quadratic sum of all contributions.
% TODO: Insert total noise equation here

Understanding which noise source dominates under given observational conditions is essential for optimizing exposure times and for interpreting the measured data \cite{Martinez2001, Moore2014}.

%------------------------------------------------------------

\section{Calibration Frames}

Calibration frames are required to correct instrumental effects present in CCD data. These frames are essential for separating the true astronomical signal from detector- and instrument-related contributions. The most commonly used calibration frames are bias frames, dark frames, and flat-field frames.

Bias frames are zero-exposure images used to measure the electronic offset and read noise of the detector. By combining multiple bias frames, a master bias frame can be constructed and subtracted from all other images to remove this electronic offset \cite{Moore2014}.

Dark frames are acquired with the same exposure time and detector temperature as the science observations but without incident light. They are used to characterize dark current and fixed-pattern effects such as hot pixels. As discussed in \cite{Martinez2001}, these effects can be effectively removed by subtracting a master dark frame.

Flat-field frames are used to correct for pixel-to-pixel sensitivity variations and large-scale illumination gradients. Perfectly uniform illumination is difficult to achieve in practice, particularly in fiber-fed spectrographs where variations in illumination can occur. As stated in \cite{Martinez2001}, these effects must be carefully accounted for to avoid introducing systematic errors into the calibrated spectra.

%------------------------------------------------------------

\section*{Acknowledgment}

The authors would like to thank the Astronomical Institute of the Czech Academy of Sciences for providing access to the Ondřejov Observatory facilities and for their support during the measurements.

\section*{References}

\bibliographystyle{unsrt}
\bibliography{bibliography}

\end{document}
