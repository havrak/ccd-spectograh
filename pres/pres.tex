\documentclass[aspectratio=43]{beamer}
\usepackage{tikz}
\usepackage{graphicx}
\usepackage{caption}

\usetheme{Berlin}
\usecolortheme{dolphin}

\beamertemplatenavigationsymbolsempty
\setbeamertemplate{footline}{%
\hfill\textbf{\insertframenumber{}/\inserttotalframenumber} \hspace{0.01cm} \vspace{0.1cm}}

\title{Characterization of Coudé CCD Spectrograph}
\author{Kryštof Havránek}
\institute{Czech Technical University in Prague \\ Astronomical Institute of the CAS}
\date{January 2026}

\begin{document}

\begin{frame}[plain]
  \maketitle
\end{frame}

\begin{frame}
  \frametitle{Overview}
  \tableofcontents
\end{frame}

% ---------------------------------------------------------
% 1. INTRODUCTION
% ---------------------------------------------------------
\section{Introduction}

\begin{frame}
  \frametitle{Instrument Overview}
  \begin{itemize}
    \item \textbf{Location:} Perek 2-meter telescope, Ondřejov Observatory.
    \item \textbf{Setup:} Coudé Spectrograph (700 mm focal length).
    \item \textbf{Detector:} PyLoN eXcelon CCD (E2V 42-10 BX).
    \item \textbf{Cooling:} Liquid nitrogen ($-115^\circ$C).
    \item \textbf{Goal:} Characterize noise, uniformity, and gain for data reduction.
  \end{itemize}
\end{frame}

\begin{frame}
  \frametitle{Spectrograph Optical Layout}
  \begin{figure}
    \centering
    \includegraphics[width=0.7\textwidth]{../img/schema_coude_spec.png}
  \end{figure}
\end{frame}

% ---------------------------------------------------------
% 2. ZERO FRAMES
% ---------------------------------------------------------
\section{Zero Frames}

\begin{frame}
  \frametitle{Zero (Bias) Frame Analysis}
  \begin{itemize}
    \item \textbf{Data:} 15 frames at shortest exposure.
    \item \textbf{Processing:} Created master zero frame using median combination, run basic statistic.
		\item \textbf{Objective:} Analyze read noise, fixed-pattern.
		\item \textbf{Statistics:}
      \begin{itemize}
        \item Master Mean: 600.15 ADU.
        \item \textbf{Read Noise:} 3.62 ADU (calculated from stack).
        \item Fixed Pattern: Negligible ($\sigma \approx 1.18$ ADU in master).
      \end{itemize}
  \end{itemize}
\end{frame}

\begin{frame}
  \frametitle{Read Noise Distribution}
  \begin{figure}
    \centering
    \includegraphics[width=0.55\textwidth]{../img/zero_stack_uniformity.png}
    \caption{Histogram of bias frames follows expected Gaussian distribution.}
  \end{figure}
\end{frame}

% ---------------------------------------------------------
% 3. DARK FRAMES
% ---------------------------------------------------------
\section{Dark Frames}

\begin{frame}
  \frametitle{Dark Frame Analysis}
  \begin{itemize}
    \item \textbf{Data:} 4 frames $\times$ 1 hour exposure.
    \item \textbf{Processing:} Subtract master zero, run basic statistic.
		\item \textbf{Objective:} Analyze dark current noise, map hot pixels.
    \item \textbf{Dark Current:}
      \begin{itemize}
        \item Value: $0.255 \times 10^{-3}~\mathrm{e}^{-}\mathrm{s}^{-1}\mathrm{px}^{-1}$.
				\item Noise at 1 hour comparable to read noise
      \end{itemize}
    \item \textbf{Hot Pixels:}
      \begin{itemize}
        \item Exhibit high values over mean in all images.
        \item Threshold: $> 5\sigma$ of master dark.
        \item Count: 252 pixels (0.024\% of sensor).
        \item Distribution: Mostly isolated (177 pixels)
      \end{itemize}
  \end{itemize}
\end{frame}

\begin{frame}
  \frametitle{Hot Pixel Map}
  \begin{figure}
    \centering
    \includegraphics[width=0.8\textwidth]{../img/dark_hot_pixels_big.png}
    \caption{Map of hot pixels (dilated).}
  \end{figure}
\end{frame}

\begin{frame}
  \frametitle{Dark Current Distribution}
\begin{figure}[htbp]
	\begin{itemize}
		\item Separation from read noise is impossible due to similar level
		\item CLT doesn't apply here - Can't model just as Gaussian.
	\end{itemize}
	\begin{center}
		\includegraphics[width=0.5\textwidth]{../img/dark_current_distribution.png}
	\end{center}
\caption{Dark Current Distribution}
\end{figure}
\end{frame}


% ---------------------------------------------------------
% 4. FLAT FIELDS
% ---------------------------------------------------------
\section{Flat Fields}

\begin{frame}
  \frametitle{Gain Calculation: Problem \& Dataset}
  \begin{itemize}
    \item \textbf{Dataset:}
      \begin{itemize}
        \item Source: Ambient illumination in the telescope dome.
        \item Exposures: Series of frames with varying exposure times.
        \item Count: 60+ frames covering whore dynamic range
      \end{itemize}
    \item \textbf{Processing:} Shot noise analysis -- calculating gain using:
      \begin{equation}
        \sigma^2 = \frac{\overline{x}}{g} + \frac{\sigma_\mathrm{ro}^2}{g^2}.
        \label{eq:numer_gain}
      \end{equation}
    \item \textbf{The Constraint:}
      \begin{itemize}
        \item Spectrograph cannot be uniformly illuminated.
				\item Less data for higher signal values.
      \end{itemize}
  \end{itemize}
\end{frame}

\begin{frame}
  \frametitle{Failed Method 1: Inter-frame Analysis}
  \begin{columns}
    \begin{column}{0.5\textwidth}
      \begin{itemize}
        \item \textbf{Method:} Variance between frames of identical exposures.
				\item \textbf{Result:} Failed due to lack of high-signal data ($< 6000~\mathrm{ADU}$).
      \end{itemize}
    \end{column}
    \begin{column}{0.5\textwidth}
      \begin{figure}
        \centering
        \includegraphics[width=\textwidth]{../img/flat_interframe_robust.png}
        \caption{PTC showing non-linear trends at low ADU.}
      \end{figure}
    \end{column}
  \end{columns}
\end{frame}

\begin{frame}
  \frametitle{Failed Method 2: Spatial Analysis}
  \begin{columns}
    \begin{column}{0.5\textwidth}
      \begin{itemize}
        \item \textbf{Method:} Calculating variance within small sliding windows (spatial statistics).
        \item \textbf{Result:} Variance changed drastically with bin size due to fiber profile (non-uniform illumination).
      \end{itemize}
    \end{column}
    \begin{column}{0.5\textwidth}
      \begin{figure}
        \centering
        \includegraphics[width=\textwidth]{../img/flat_spatial_robust.png}
        \caption{Spatial Photon Transfer Curve, incoherent variance vs signal.}
      \end{figure}
    \end{column}
  \end{columns}
\end{frame}

\begin{frame}
  \frametitle{Successful Method?: Residual Analysis}
  \begin{itemize}
    \item \textbf{Concept:} Compare data against a "noise-free" model.
    \item \textbf{Workflow:}
      \begin{enumerate}
        \item \textbf{Model:} Create mean normalized model from all flats.
        \item \textbf{Scaling:} Scale model to match flux of each frame.
        \item \textbf{Residuals:} Calculate $\sigma^2 = Data - Model$.
				\item \textbf{Stratified Sampling}: Sample randomly within level bins.
        \item \textbf{Rejection:} Median Absolute Deviation (MAD) to remove outliers.
				\item \textbf{Function fitting:} Fit in mean vs sigma dependence to estimate gain.
      \end{enumerate}
  \end{itemize}
\end{frame}

\begin{frame}
  \frametitle{Residual Method: 10s Model}
  \begin{itemize}
    \item \textbf{Approach:} Constructing the model using pictures of a single exposure
  \end{itemize}
\begin{figure}[htbp]
	\begin{center}
		\includegraphics[width=0.5\textwidth]{../img/model_10s_photon_transfer_curve.png}
	\end{center}
\caption{Photon Transfer Curve -- Residual Analysis, 10~s Exposure Model}
\label{img:flat_model_10s}
\end{figure}
\end{frame}

\begin{frame}
  \frametitle{Residual Method: 10s Model}
  \begin{figure}
    \centering
    \includegraphics[width=0.6\textwidth]{../img/model_10s_histogram_med.png}
    \caption{Histogram of residuals using 10s model}
  \end{figure}
\end{frame}

\begin{frame}
  \frametitle{Residual Method: Global Model}
  \begin{itemize}
    \item \textbf{Approach:} Constructing the model using \textbf{all} flat frames.
		\item  Gain $g = 0.6114 \pm 0.0045~\mathrm{e}^{-}\mathrm{ADU}^{-1}$
  \end{itemize}
\begin{figure}[htbp]
	\begin{center}
		\includegraphics[width=0.5\textwidth]{../img/model_photon_transfer_curve.png}
	\end{center}
\caption{Photon Transfer Curve -- Residual Analysis}
\end{figure}
\end{frame}

% ---------------------------------------------------------
% 5. CONCLUSION
% ---------------------------------------------------------
\section{Conclusion}

\begin{frame}
  \frametitle{ThAr \& Conclusion}
  \begin{itemize}
    \item \textbf{ThAr Verification:}
      \begin{itemize}
        \item Compared lamp spectra against atomic atlases.
        \item Verified grating resolution matches documentation.
      \end{itemize}
    \item \textbf{Final Characterization:}
      \begin{itemize}
        \item \textbf{Read Noise:} 3.62 ADU (Low).
        \item \textbf{Dark Current:} Negligible for standard ops.
        \item \textbf{Gain:} $\approx 0.61~\mathrm{e}^{-}\mathrm{ADU}^{-1}$.
      \end{itemize}
    \item \textbf{Verdict:} System is exhibits low noise, behaves uniformly, gain analysis is trouble due to nature of the dataset.
  \end{itemize}
\end{frame}

\end{document}
