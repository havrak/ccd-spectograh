\documentclass[conference]{IEEEtran}
\IEEEoverridecommandlockouts

\usepackage{cite}
\usepackage{amsmath,amssymb,amsfonts}
\usepackage{algorithmic}
\usepackage{graphicx}
\usepackage{textcomp}
\usepackage{xcolor}


\def\BibTeX{{\rm B\kern-.05em{\sc i\kern-.025em b}\kern-.08em
T\kern-.1667em\lower.7ex\hbox{E}\kern-.125emX}}

\begin{document}

\title{Measuring Properties of a CCD Spectrograph}

\author{\IEEEauthorblockN{1\textsuperscript{st} Kryštof Havránek}
\IEEEauthorblockA{\textit{Course: Space Engineering \the\year} \\
\textit{Czech Technical University in Prague}\\
Technicka 2, Prague, Czech Republic \\
havrakry@fel.cvut.cz}
\and
\IEEEauthorblockN{2\textsuperscript{nd} Vega Grau Esteve}
\IEEEauthorblockA{\textit{Course: Space Engineering \the\year} \\
\textit{Politechnical University of Valencia}\\
Camino de Vera 46022, Valencia, Spain \\
vega03upv@gmail.com}
\and
\IEEEauthorblockN{Mentor}
\IEEEauthorblockA{\textit{Course: Space Engineering \the\year} \\
\textit{Astronomical Institute of the CAS}\\
Fričova 298, Ondřejov, Czech Republic \\
martin.jelinek@asu.cas.cz}
}

\maketitle

\begin{abstract}
Following paper concerns itself with a process of measuring and calculating parameters of a Cloude CCD spectrograph connected to a Perek 2 meter optical telescope. Measured properties were readout noise, dark current behavior and lastly gain.
\end{abstract}

%------------------------------------------------------------

\section{Introduction}

One of the primary tools used to investigate the physical properties of celestial objects is astronomical spectroscopy. This carries out a process that enables the determination of stellar temperatures, chemical composition, radial velocities and other fundamental parameters by dispersing incoming light into its constituent wavelengths. Nowadays, modern spectroscopic observations rely on large-aperture telescopes which are combined with sensitive CCD detectors and stable spectrographs.

As emphasized by Martinez et al.\ in \textit{A Practical Guide to CCD Astronomy}, the scientific quality of CCD-based observations depends not only on the optical system but also on a precise understanding of detector behavior and noise sources, as well as on proper calibration procedures \cite{Martinez2001}. Similarly, \textit{Lessons from the Masters} highlights that accurate spectroscopic measurements require careful instrument characterization and calibration before any scientific interpretation can be performed \cite{Moore2014}.

This protocol describes the instrumental setup and theoretical background relevant to spectroscopic observations carried out with the Ondřejov 2-m telescope. Particular emphasis is placed on the principles of spectrograph operation, CCD detector characteristics, and the main noise sources affecting the acquired data. This theoretical foundation provides the basis for the data reduction and analysis steps described in later sections.

%------------------------------------------------------------

\section{Ondřejov 2-m Telescope}

The Ondřejov 2-m telescope is built as a classical Cassegrain reflector with a primary mirror which has a diameter of two meters and is operated by the Astronomical Institute of the Czech Academy of Sciences. This sort of telescopes provide a compromise between mechanical stability and high light-gathering power, making them suitable for high-resolution spectroscopic observations of relatively faint targets.

In a Cassegrain optical configuration, the light that is collected by the primary mirror is reflected towards a secondary mirror and then redirected to the focal plane through a central aperture. This design makes it possible for spectrographs and auxiliary instrumentation to maintain a compact optical layout while they are mounted in a stable position. As discussed in \cite{Martinez2001}, these configurations are advantageous for spectroscopy as mechanical flexure and optical stability directly affect wavelength calibration accuracy and spectral resolution.

This telescope is equipped with accurate guiding and tracking systems, which are essential for long spectroscopic exposures. Any guiding errors during an exposure may result in spectral line broadening or reduced signal-to-noise ratio, especially when narrow entrance slits or optical fibers are used \cite{Moore2014}.

%------------------------------------------------------------

\section{Spectrograph Overview and Optical Layout}

A spectrograph is an optical instrument designed so that incoming light is dispersed into its different wavelength components. In astronomical spectrographs, this dispersion is achieved by using a diffraction grating. According to \cite{Moore2014}, the basic components required for a spectrograph to function correctly are an entrance slit or optical fiber, a collimator, a dispersive element, a camera system, and a detector.

The working mechanism of the spectrograph begins with the light entering through the slit or optical fiber, where it is collimated into a parallel beam. This beam encounters the diffraction grating, which separates the light according to wavelength. The dispersed light is then focused onto the CCD detector by camera optics, where the spectrum is recorded. As mentioned in \cite{Martinez2001}, fiber-fed spectrographs are often preferred because they decouple the spectrograph mechanically from the telescope, reducing flexure and improving long-term stability.

However, fiber-fed systems introduce additional considerations. Variations in fiber positioning or illumination can lead to changes in intensity across the detector, which must be corrected during flat-field calibration. As stated in \cite{Martinez2001}, imperfect illumination of the spectrograph entrance can imprint systematic patterns onto the recorded spectra, highlighting the importance of careful flat-fielding and illumination control.

%------------------------------------------------------------

\section{Spectral Orders and Wavelength Calibration}

High-resolution spectrographs often operate in multiple diffraction orders, particularly when echelle gratings are used. In this system, each spectral order corresponds to a different diffraction condition of the grating, allowing a wide wavelength range to be recorded simultaneously on the CCD detector. As described in \cite{Moore2014}, overlapping orders are separated using a cross-dispersing element, producing a two-dimensional spectral format.

Accurate wavelength calibration is essential for spectroscopic analysis. This is typically achieved using emission-line lamps, such as Thorium--Argon (ThAr) lamps, which provide a dense set of narrow spectral lines with well-known wavelengths. As described in \cite{Martinez2001}, ThAr lamps are particularly suitable for high-precision wavelength calibration due to their stability and line density.

While the detailed matching of observed ThAr spectra to reference line lists is part of the data reduction process, understanding the underlying calibration principle is essential when designing the observational protocol and interpreting the results.

%------------------------------------------------------------

\section{CCD Detector Fundamentals}

Charge-coupled devices are widely used in astronomical spectroscopy due to their high quantum efficiency, linear response over a wide dynamic range, and relatively low noise characteristics. These properties make CCDs suitable for the detection of faint astronomical signals and for applications requiring precise spectroscopic and photometric measurements.

Incident photons interact with the silicon substrate of the CCD, generating photoelectrons through the photoelectric effect. The number of generated electrons is proportional to the incident photon flux and the detector quantum efficiency. The photoelectrons are accumulated in potential wells associated with individual pixels during the exposure time. Through a sequence of clocked voltage shifts, the stored charge is transferred across the detector and eventually read out at an amplifier.

During readout, the collected charge is converted into a voltage signal and digitized by an analog-to-digital converter, producing values recorded in Analog-to-Digital Units (ADU). This process introduces several instrumental effects that must be characterized and corrected during data reduction \cite{Moore2014}.

The CCD gain determines the conversion between the number of electrons and the recorded ADU value. Accurate knowledge of the gain is essential for converting measured signals into physically meaningful quantities and for estimating noise contributions. As emphasized in \cite{Martinez2001}, an incorrect gain value leads to systematic errors in noise estimation and signal-to-noise ratio calculations.

Another important characteristic of CCD detectors is their linear response, which ensures that the recorded signal is proportional to the number of incident photons over a wide range of illumination levels. Deviations from linearity may occur at very low signal levels or near full-well capacity due to charge saturation. Linearity tests are therefore commonly performed to determine the valid operational range of the detector \cite{Martinez2001}.

%------------------------------------------------------------

\section{Noise Sources in CCD Detectors}

Several independent noise sources affect measurements performed with CCD detectors. The main contributions are read noise, dark current noise, and photon (shot) noise.

Read noise is introduced during the electronic readout process and is independent of exposure time. It originates from multiple electronic components such as the output amplifier, charge-to-voltage conversion stage, and analog-to-digital converter. Fluctuations during readout lead to uncertainty in the measured signal, even in the absence of incident light. Read noise is typically characterized using bias frames and dominates at low signal levels \cite{Moore2014}.

Dark current noise arises from thermally generated electrons within the silicon lattice of the CCD. These electrons are indistinguishable from photoelectrons and accumulate linearly with exposure time. Since the generation of thermal electrons is a random process, the associated dark noise follows Poisson statistics.
% TODO: Insert dark noise equation here

As stated in \cite{Martinez2001}, dark current strongly depends on detector temperature and can increase rapidly with small temperature variations. CCD detectors are therefore commonly cooled to reduce dark current and its associated noise.

Photon noise, also known as shot noise, originates from the statistical nature of photon arrival at the detector. Even for a constant light source, the number of detected photons fluctuates according to Poisson statistics, introducing an uncertainty that cannot be removed by calibration.
% TODO: Insert photon noise equation here

Assuming that the noise sources are statistically independent, the total noise in a CCD pixel is given by the quadratic sum of all contributions.
% TODO: Insert total noise equation here

Understanding which noise source dominates under given observational conditions is essential for optimizing exposure times and for interpreting the measured data \cite{Martinez2001, Moore2014}.

%------------------------------------------------------------

\section{Calibration Frames}

Calibration frames are required to correct instrumental effects present in CCD data. These frames are essential for separating the true astronomical signal from detector- and instrument-related contributions. The most commonly used calibration frames are bias frames, dark frames, and flat-field frames.

Bias frames, or zeros, are zero-exposure images used to measure the electronic offset and read noise of the detector. By combining multiple bias frames, a master bias frame can be constructed and subtracted from all other images to remove this electronic offset \cite{Moore2014}.

Dark frames are acquired with the same exposure time and detector temperature as the science observations but without incident light. They are used to characterize dark current and fixed-pattern effects such as hot pixels. As discussed in \cite{Martinez2001}, these effects can be effectively removed by subtracting a master dark frame.

Flat-field frames are used to correct for pixel-to-pixel sensitivity variations and large-scale illumination gradients. Perfectly uniform illumination is difficult to achieve in practice, particularly in fiber-fed spectrographs where variations in illumination can occur. As stated in \cite{Martinez2001}, these effects must be carefully accounted for to avoid introducing systematic errors into the calibrated spectra.

%------------------------------------------------------------


\section{Bias Frame Analysis}

15 bias frames were taken at lowest exposition time possible.
Exact time in seconds is not known as camera limitations induce some static delay.
Later on from dark frames it was estimated to be around half a second, but for the purpose of analyzing zero frames it doesn't play a role and neither was a lot of attention given to making the offset estimate particularly accurate.

Analysis of bias frames from the spectrograph is relatively straightforward.
First of all median frame was created, to remove any fixed-pattern CCD might exhibit.
Median is used to prevent master frame from being influenced by any high energy particles that might impact the sensor during short observation and sway the results.
For master frame and stack of all bias frames basic statics were calculated as showed in table \ref{tab:zero}.

\begin{table}[htbp]
\caption{Bias Frame Analysis Results}
\begin{center}
\begin{tabular}{|c||c|c|}
\hline
	\textbf{Data} & $\mu$ (ADU) & $\sigma$ (ADU) \\
\hline
\hline
	Master frame & 600.15 & 1.18 \\
\hline
	All bias frames & 500.14 & 3.62 \\
\hline
\end{tabular}
	\label{tab:zero}
\end{center}
\end{table}

We can see that the readout noise is fairly small and also from really low deviation within master frame that the sensor doesn't exhibit any noticeable fixed-pattern (also apparent from \ref{img:master_bias}).
Bellow are images, one for statics calculated within master frame \ref{img:master_zero}, second for all zeros taken \ref{img:zero_stack}.
As some high energy particles impacted the sensor in some frames second histogram limited to at max 625~ADU.
Still it is clearly visible that the measured noise has expected Gaussian distribution.

\begin{figure}[htbp]
	\begin{center}
		\includegraphics[width=0.48\textwidth]{../img/zero_master_frame.png}
	\end{center}
	\caption{Master Zero Frame}
\label{img:master_bias}
\end{figure}

\begin{figure}[htbp]
	\begin{center}
		\includegraphics[width=0.4\textwidth]{../img/zero_master_uniformity.png}
	\end{center}
\caption{Master Zero Frame Uniformity}
\label{img:master_zero}
\end{figure}

\begin{figure}[htbp]
	\begin{center}
		\includegraphics[width=0.4\textwidth]{../img/zero_stack_uniformity.png}
	\end{center}
\caption{Total Zero Stack Noise Distribution}
\label{img:zero_stack}
\end{figure}

\section{Dark Frame Analysis}

All together four dark frames were taken each with one hour long exposition.
At first master zero frame was subtracted from all dark frames, to remove any fixed-pattern, regardless how small.
Multiple different properties were then observed.
Those were dark current characteristics, number of hot pixels and transient events (which needed to be filtered out).

A threshold for hot pixels was set to be a values higher than 5 times the standard deviation of the master dark frame, according to \cite{KennedyYoung2020}.
In individual dark frames it would be impossible to distinguish between high energy particles impacting the sensor and hot pixels thus this analysis was only run on the master dark frame, since hot pixels should exhibit values significantly higher then mean in all images.
For threshold of 5 times the standard deviation 252 such pixels were found, this comprises 0.024~\% of the area of the CCD sensor.
Were we to use much more conservative threshold of 10 times the std. only 17 pixels meet this criteria.

Some of these hot pixels fall within areas that are illuminated by the light coming from the dispersive grating.
However most pixels (177 or 69.4~\%) are isolated and don't form any larger cluster, so their impact on measurement should be negligible.
Remaining 77 have another hot pixels in their vicinity (checked by doing 2D convolution using 5x5 matrix of ones).
With largest groups having as many as 10 pixels close to each other, which could possibly be a small manufacturing defect.
Their locations can be seen on picture \ref{img:hot_pixels}, each dot has a size of 30 pixels as to be visible.

\begin{figure}[htbp]
	\begin{center}
		\includegraphics[width=0.48\textwidth]{../img/dark_hot_pixels_big.png}
	\end{center}
\caption{Hot Pixels Locations}
\label{img:hot_pixels}
\end{figure}


Aside from constantly hot pixels transient events were also detected.
As already stated, while hot pixels will exhibit high values on all dark frames, pixels effected by transient event (cosmic rays, background radiation) will be hot only on single frame.
Taking a maximal value for each pixel from all darks and comparing it to median dark frame, some 707 pixels were over a threshold of 1000 ADU.

As for statistical properties of the dark frames.
Master dark frame exhibits a low standard deviation of 2.01~ADU, with average values of 1.89~ADU.
Deviation calculated over all frames is 6.77~ADU, when transient events are excluded this drops to 6.02~ADU.
Verifying if the measured data follow expected distribution is rather difficult.
Shot noise from dark currents should exhibit Poisson distribution, however as it's variance is quite small -- that is comparable with one for readout noise -- it's difficult to separate both noises.
While variance of read noise and master zero is known, this noise is still present in darks and cannot be just subtracted.

In order to express dark currents in accurate units $e^{-}s^{-1}$ gain value of $0.6727\mathrm{e}^{-}\mathrm{ADU}^{-1}$, estimated in following chapters, was used.
Using this gain mean dark current was $0.328\cdot 10^{-3}~\mathrm{e}^{-}\mathrm{s}^{-1}\mathrm{px}^{-1}$ with deviation of $0.3487\cdot 10^{-3}~\mathrm{e}^{-}\mathrm{s}^{-1}$
Distribution of dark current overlaid with Poisson distribution curve for calculated $\mu$ and $\sigma$ is showed on picture \ref{img:dark_distribution}.

\begin{figure}[htbp]
	\begin{center}
		\includegraphics[width=0.4\textwidth]{../img/dark_current_distribution.png}
	\end{center}
\caption{Dark Current Distribution}
\label{img:dark_distribution}
\end{figure}

\section{Gain Calculation}

In order to estimate gain 60 different images with different exposure times and at four different illumination intensities were taken.
Light source was ambient light in the telescope dome.
Different light intensities were achieved by offsetting optical fibres that bring light to the spectrograph relative to each other.
Thus no prior information on how much did different intensities differ was known.

Basic idea for noise calculation stems from the fact that arriving photons exhibit a Poisson distribution.
Normally for Poisson distribution mean value and variance are equal to each other.
However when amplification/attenuation is present this doesn't hold true and gain value needs to be included into equation.
If we also add influence of other noises (here just read noise $\sigma_\mathrm{ro}^2$) we get an equation
\begin{equation}
	g\cdot \sigma = \sqrt{\overline{x}\cdot g + \sigma_\mathrm{ro}^2}.
	\label{eq:gain_eq}
\end{equation}
As the noise is statistically independent it's simply included in the sum, $\overline{x}$ is mean signal value, $\sigma$ it's standard deviation and lastly $g$ is gain\cite{Newberry1996}.

This equation is often modified for the purpose of numerical calculation as
\begin{equation}
	\sigma^2 = \frac{\overline{x}}{g} + \frac{\sigma_\mathrm{ro}^2}{g}.
\end{equation}
This removes unnecessary step of calculating square root.

In non where sensor installation can enable a uniforma flatfields illumination good gain estimate can be made from just a couple of images\cite{irafGain}.
Variances can also be calculated easily across different pixels - if their properties are similar.
Coulde spectrograph in Ondřejov however doesn't allow an easy modification to the structure to ensure uniform illumination of the sensor.
Incoming light thus still comes through the spectral dispersive grating and different parts of spectrograph are then illuminated with different light intensities and at different wavelength.
As the light source doesn't have an constant spectrum illumination will thus be strictly uniform.
In addition but a small part of the sensor is actually illuminated -- there are two thin strips each for one of two optical fibres brining light to the sensor, while most of the sensor area is unilluminated.

\subsection{Inter-Frame Flatfield Analysis}

First tested approach to calculate gain was based on calculating variances and means for groups of images of a same exposure time.
Variances were thus calculated not over groups of multiple pixels but for each pixel independently as guided by \cite{Robertson_2021}.
Images, after subtraction of master zero frame, were first group by same illumination level and exposure.
Then for groups where more than one image was present statistical properties were calculated.

While for transient event detection in dark frames simple outlier rejection method based on $\overline{x} + n\cdot \sigma$ threshold was deemed sufficient here a more robust rejection method was implemented.
This was done as in individual bins there might not be that many points and standard deviation might become significantly influenced by outliers -- thus not rejecting them properly.
Instead a median absolute value (MAD) was used.
MAD relies on median, which is much less likely to be influenced by outliers, otherwise the basic principle is similar to classical thresholding.
Mean values of the data are sufficiently large for central limit theorem to apply \cite{Walpolec2002}.
Thus a classical MAD scaling value of 1.4826 was kept, while cutoff value was chosen empirically based on data histogram.

Problem of this approach was that dataset contains multiple images for the same configuration only for low exposure times (bellow 10 seconds).
Thus only values under 6000 ADU were accounted for in the gain calculation \ref{img:interframerobust}.
This was deemed to be insufficient as peak value of the 16~bit is more than ten times larger.

\begin{figure}[htbp]
	\begin{center}
		\includegraphics[width=0.4\textwidth]{../img/flat_interframe_robust.png}
	\end{center}
\caption{Photon Transfer Curve -- Inter-Frame Analysis}
\label{img:interframerobust}
\end{figure}

Gain value of $g=0.1915$ was calculated using this method.
Due to problematic nature of this approach it's confidence intervals weren't calculated and read noise wasn't set to scale with $\frac{1}{g^2}$.
Still it's apparent that the sensor doesn't exhibit expected behavior and thus it's gain isn't completely linear.
There's especially a visible trend of median deviator values at low ADU levels increasing much faster than they should.
Or it's also possible that fibre alignment wasn't as steady as it should be and spectral lines were slightly shifted between images.


\subsection{Spatial Flatfield Analysis}

As at least pixels properties a are relatively constant in the sensor it was attempted to estimate gain while calculating statistical properties within individual pictures.
In X axis are separated different wavelength it wouldn't be correct to group pixels in this direction.
In Y direction however only one spectral line is present, however still not whole sensor is illuminated by it and even within bands with spectral lines brightness level isn't constant.
Still were the gain values constant across different bin sizes it would provide some useful data.

However as shown on table \ref{tab:spatial_flat} gain value changes significantly with bin sizes.
And even when visualizing binned data it's clear that the trend isn't there \ref{img:flat_spatial_robust.png}

\begin{table}[htbp]
\caption{Spatial Flatfield Analysis Gain Estimates}
\begin{center}
\begin{tabular}{|c||c|c|}
\hline
	 Bin Height (px) & Gain ($\mathrm{e}^{-}/\mathrm{ADU}$) \\
\hline
\hline
2 & 0.1211 \\
\hline
3 & 0.0854 \\
\hline
4 & 0.0826 \\
\hline
5 & 0.0401 \\
\hline
\end{tabular}
	\label{tab:spatial_flat}
\end{center}
\end{table}

\begin{figure}[htbp]
	\begin{center}
		\includegraphics[width=0.4\textwidth]{../img/flat_spatial_robust.png}
	\end{center}
\caption{Photon Transfer Curve -- Spatial Analysis}
\label{img:flat_spatial_robust}
\end{figure}

\subsection{Residual Flatfield Analysis}




\section*{Acknowledgment}

The authors would like to thank the Astronomical Institute of the Czech Academy of Sciences for providing access to the Ondřejov Observatory facilities and for their support during the measurements.

\bibliographystyle{unsrt}
\bibliography{bibliography}

\end{document}
